\documentclass[a4paper,12pt]{article}

\usepackage{cmap}					% поиск в PDF
\usepackage[T2A]{fontenc}			% кодировка
\usepackage[utf8]{inputenc}			% кодировка исходного текста
\usepackage[english,russian]{babel}	% локализация и переносы

\author{Астраханцев Роман Геннадьевич}
\title{ОТЧЕТ о прохождении производственной практики}
\date{2021 г.}

\begin{document} % Конец преамбулы, начало текста.
	
	\maketitle
	
	\section*{ВВЕДЕНИЕ}
	
	В рамках производственной практики, проходящей c 05.07.2021 по 31.07.2021 на кафедре компьютерной безопасности, перед студентом стояла задача описания принципиальной работы алгоритма пороговой схемы на основе подписи Эль-Гамаля. 
	
	В частности, студенту предстояло разобраться в задаче выработке валидной подписи несколькими участниками, которая бы проверялась одним публичным ключом. 
	
	В качестве дополнительных задач студенту предлагалось описать принципиальную схему выработки такой подписи с нулевым доверием между участниками, а также проанализировать полученную схему на предмет защищённости от раскрытия секретов злоумышленными участниками. 
	
	Пороговые схемы становятся популярным инструментом для создания межблокчейнового взаимодействия, поэтому задача построения защищённых пороговых схем становится всё более актуальной.
	
	
	\section{Задачи пороговой подписи }	
	
	С появлением смарт-контрактов в блокчейн сетях появилась возможность принятия коллегиального решения разными сообществами. Наивной реализацией принятия такого решения было сочетание смарт-контрактов и мультиподписи. Процесс вынесения определённого решения заключался в создании смарт-контракта, который исполнялся тогда и только тогда, когда заданное количество собранных  электронных подписей будет аккумулятивно подано на вход смарт-контракту (возможно за несколько транзакций). 
	
	Такой подход имеет ряд недостатков:
	
	1)	На каждое новое решение требуется создание нового одноразового смарт-контракта;
	
	2)	Проверка коллегиального решения требует проверки каждой поданной подписи;
	
	3)	Сбор электронных подписей требует взаимодействия с блокчейном, что в свою очередь может быть долго/ресурсозатратно.
	
	Для решения этих проблем было предложено использование технологии пороговых подписей. Идея заключается в том, чтобы вместо нескольких уникальных подписей от разных участников вырабатывать коллективную подпись, которую затем можно будет проверить коллективным публичным ключом. Пороговой такая схема называется из-за того, что в самой схеме задаётся минимальное количество участников (порог), которое необходимо для выработки коллективной подписи произвольного сообщения.
	
	Подобная схема избавлена от всех описанных выше недостатков, однако имеет сложности в своей реализации. Так, например, как вырабатывать секреты участников децентрализовано, как гарантировать, что именно минимальное количество участников сможет подписать сообщение. Эти и другие вопросы рассмотрены в этой работе.
	
	Стоит отметить, что пороговые схемы находят своё применение не только в избавлении от генерации смарт-контрактов и увеличении производительности взаимодействия с блокчейном, но в любой задаче требующей выработки единого коллективного решения. К примеру, совет директоров той или иной ком-пании может выработать единую подпись от лица самой компании без раскрытия информации о том, кто был за это решение, в то время как использование мультиподписи раскрывает эту информацию. Ещё одним применением этой технологии в современных реалиях будет возможность осуществления эффективного взаимодействия между блокчейнами или шардами одного блокчейна.
	
	
	\section{Определение пороговой схемы}
	\section{Наивная схема}
	\section{Схемы с нулевым доверием}
		\subsection{Генерация ключа}
		\subsection{Генерация подписи}


	\section*{ВЫВОДЫ}
	
	\section*{ЗАКЛЮЧЕНИЕ}
	
	\section*{БИБЛИОГРАФИЧЕСКИЙ СПИСОК}
	
	\section*{ПРИЛОЖЕНИЯ}
	
	\renewcommand\thesubsection{\arabic{subsection}}
	
		\subsection{Схема Шамира}
		\subsection{Схема электронной подписи Эль-Гамаля}
		\subsection{Умножение и инверсия секретов по их долям}
	
	
\end{document} % Конец текста.