\documentclass[a4paper,12pt]{article}

\usepackage{cmap}					% поиск в PDF
\usepackage[T2A]{fontenc}			% кодировка
\usepackage[utf8]{inputenc}			% кодировка исходного текста
\usepackage[english,russian]{babel}	% локализация и переносы
\usepackage{enumitem}  				% смена типа символя у enumerate
\usepackage{amsmath,amsfonts,amssymb,amsthm,epsfig,epstopdf,titling,url,array}
\usepackage{icomma} 				% "Умная" запятая: $0,2$ --- число, $0, 2$ --- перечисление
\usepackage{hyperref}				% кликабельные ссылки
\usepackage{soulutf8} 				% Модификаторы начертания
\usepackage{mathrsfs}				% теоремы
\usepackage{tabularx}				% X таблицы
\usepackage[backend=biber,
			bibencoding=utf8,
			bibstyle=numeric,
			sorting=none]			% цитирование в порядке появляения в тексте
		{biblatex}					% библиография
\usepackage{xparse} 				% для переопределения section* subsection* и пр., чтобы те появлялись в оглавлении
\usepackage[ruled,
			linesnumbered,
			vlined]
		{algorithm2e}				% красивые алгоритмы

\sloppy

\SetAlgorithmName{Алгоритм}{algo}{Список алгоритмов}
\SetKwInput{KwData}{Вход}
\SetKwInput{KwResult}{Выход}
\SetNlSty{textbf}{}{.}
\SetAlgoNlRelativeSize{0}
\SetKwIF{If}{ElseIf}{Else}{Если}{то}{иначе если}{иначе}{}

\addbibresource{sources.bib}

\theoremstyle{definition}
\newtheorem{defn}{Определение}


\let\oldsection\section
\makeatletter
\newcounter{@secnumdepth}
\RenewDocumentCommand{\section}{s o m}{%
	\IfBooleanTF{#1}
	{\setcounter{@secnumdepth}{\value{secnumdepth}}% Store secnumdepth
		\setcounter{secnumdepth}{0}% Print only up to \chapter numbers
		\oldsection{#3}% \section*
		\setcounter{secnumdepth}{\value{@secnumdepth}}}% Restore secnumdepth
	{\IfValueTF{#2}% \section
		{\oldsection[#2]{#3}}% \section[.]{..}
		{\oldsection{#3}}}% \section{..}
}
\makeatother

\author{Астраханцев Роман Геннадьевич}
\title{ОТЧЕТ о прохождении производственной практики}
\date{2022 г.}

\begin{document} % Конец преамбулы, начало текста.
	
	\thispagestyle{empty}
	\begin{center}
		Федеральное государственное автономное учреждение \\ высшего профессионального образования \\

		НАЦИОНАЛЬНЫЙ ИССЛЕДОВАТЕЛЬСКИЙ УНИВЕРСИТЕТ \\ <<ВЫСШАЯ ШКОЛА ЭКОНОМИКИ>> \\
		\vspace{2ex}
		Образовательная программа «Компьютерная безопасность» \\
	\end{center}

	\begin{center}
		\vspace{15ex}
		\textbf{\so{КУРСОВАЯ РАБОТА} \\ \vspace{3ex} Анализ материалов по теме \\ <<Эндоморфизмы на эллиптических кривых>>}
	\end{center}

	\vspace{17ex}

	\hspace{0.6\linewidth} 	\textbf{Выполнил:} 
	
	\hspace{0.6\linewidth}  студент группы СКБ-171 
	
	\hspace{0.6\linewidth}  Астраханцев Р.Г. 


	
	\begin{center}
		
		\vfill
		\textbf{Москва, 2022}
	\end{center}
	
	\newpage
	\tableofcontents
	
	\newpage
	\section*{ВВЕДЕНИЕ}
	
	В криптографических схемах на основе эллиптических кривых умножение точки на число является наиболее времязатратной операцией. Иными словами, вычиление кратной точки $k P$, где $k$ -- целое число, а $P$ -- точка на эллиптической кривой $E$, определённой над конечным полем $\mathbb{F}_q$, считается довольно медленной операцией. Базовой техникой вычисления этой точки служит алгоритм удвоения-и-сложения, который работает за $\log_2 k$ операций сложения на эллиптической кривой. Множество методов для ускорения произведения были предложены в литературе \cite{gordon1998survey, hankerson2000software, menezes2018handbook}, например:
	
	\begin{enumerate}
		\item Общие методы, которые позволяют ускорить вычисления в любой конеченой абелевой группе.
			\begin{enumerate}
				\item Использование таблиц предварительных вычислений \cite{lim1994more}, зависящих от базовой точки $P$. Они могут быть применимы в случая, когда $P$ фиксировано.
				\item Использование цепочек сложения, когда $k$ фиксировано.
				\item Оконные методы, когда базовая точка $P$ не известна.
				\item Одновременное использование нескольких методов для вычисления выражений вида $k_1 P_1 + \ldots + k_s P_s$
			\end{enumerate}

		\item Методы экспоненциального перекодирования, которые заменяют двоичное представление $k$ другим с меньшим количеством ненулевых знаков \cite{gollmann1996redundant}.
		
		\item Методы, характерные для умножения точек на эллиптических кривых.
			\begin{enumerate}
				\item Выбор конечного поля $\mathbb{F}_q$ таким образом, чтобы вычисления в нём происходили быстрее. Например, выбор поля $F_q$, где $q$ -- это простое число Мерсена \cite{solinas1999generalized}.
				\item Выбор представления конечного поля $\mathbb{F}_q$ таким образом, чтобы вычисления в нём происходили быстрее. Например, выбор неприводимого трехчлена в качестве примитивного многочлена для расширения бинарных полей.
				\item Выбор представления точки $P$, которая позволяет проводить опреации сложения и умножения быстрее \cite{cohen1998efficient}.
				\item Выбор эллиптических кривых со специальными свойствами, как например кривых Коблица \cite{koblitz1991cm}
			\end{enumerate}
	\end{enumerate}

	Основное приемущество кривых Коблица заключается в том, что на них можно использовать эндоморфизм Фробениуса для быстрого умножения точек. А именно, методы Солинаса \cite{solinas1997improved, solinas2000efficient} на этих кривых не используют удвоений точек вообще. Данные методы могут быть обощены на случай произвольного эндоморфизма, однако в целом это не эффективно.
	
	Тем не менее, для эллипических кривых, у которых существует эффективно-вычислимый эндоморфизм можно добиться увеличения скорости вычисления кратной точки путём декомпозиции числа $k$ по некторому модулю $n$ на составные части и применения одного преобразования эндоморфзма. Хотя для кривых Коблица данный метод уступает по скорости методам Солинаса, он всё ещё полезен на более широком классе кривых (в частности, кривых, определённых над $\mathbb{F}_q$, где $q$ -- простое число). Кроме того, такой подход позволяет добиться ускорения даже в случае, когда преобразование эндомофризма дороже, чем операции над точками.
	
	\newpage
	
	\section{Идея применения эффектиных эндоморфизмов}
		
	\section{Примеры эффективных эндоморфизмов}

	\section{Декомпозиция $k$}

	\section*{ЗАКЛЮЧЕНИЕ}
	
	
	\newpage
	\printbibliography[title={БИБЛИОГРАФИЧЕСКИЙ СПИСОК}]
		
	
\end{document} % Конец текста.