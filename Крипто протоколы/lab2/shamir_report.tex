\documentclass[a4paper,12pt]{article}

\usepackage{cmap}					% поиск в PDF
\usepackage[T2A]{fontenc}			% кодировка
\usepackage[utf8]{inputenc}			% кодировка исходного текста
\usepackage[english,russian]{babel}	% локализация и переносы
\usepackage{enumitem}  				% смена типа символя у enumerate
\usepackage{amsmath,amsfonts,amssymb,amsthm,epsfig,epstopdf,titling,url,array}
\usepackage{icomma} 				% "Умная" запятая: $0,2$ --- число, $0, 2$ --- перечисление
\usepackage{hyperref}				% кликабельные ссылки
\usepackage{soulutf8} 				% Модификаторы начертания
\usepackage{listings}				% Листинг исходного кода
\usepackage{array,tabularx,tabulary,booktabs} % Дополнительная работа с таблицами

\usepackage{xcolor}

\definecolor{codegreen}{rgb}{0,0.6,0}
\definecolor{codegray}{rgb}{0.5,0.5,0.5}
\definecolor{codepurple}{rgb}{0.58,0,0.82}
\definecolor{backcolour}{rgb}{0.95,0.95,0.92}

\lstdefinestyle{mystyle}{
	backgroundcolor=\color{backcolour},   
	commentstyle=\color{codegreen},
	keywordstyle=\color{magenta},
	numberstyle=\tiny\color{codegray},
	stringstyle=\color{codepurple},
	basicstyle=\ttfamily\footnotesize,
	breakatwhitespace=false,         
	breaklines=true,                 
	captionpos=b,                    
	keepspaces=true,                 
	numbers=left,                    
	numbersep=5pt,                  
	showspaces=false,                
	showstringspaces=false,
	showtabs=false,                  
	tabsize=2
}

\lstset{style=mystyle}

\allowdisplaybreaks

\sloppy

\title{Реализация Схемы Блома распределения ключей }
\author{Роман Астраханцев, СКБ-171}

\begin{document}
	\maketitle
	
	\section{Написание программы}
	Была написана программа на языке программирования Python, которая реализует схему разделения секрета Щамира для $n$ участников, при которой обеспечивается востановление секрета любыми $t$ участниками, $1\le t \le n$. Величины $p, n, t$ задаются пользователем с консоли, $r_1~=~1, \dots, r_n~=~n$. Коэффициенты $a_i$ исходного многочлена $f(x) = \sum_{i=0}^{t-1} a_{i} x^i$ инициализируются случайно (как элементы $\mathbb{Z}_p$) при запуске программы. Текст программы может быть найден в приложении~\ref{program}.
	
	\section{Расчёт ключевых значений}
	
	Студент выполнял вариант № k=4.

	\subsection{p=73, t=7, n=15}
	
	\[ f(x) = 17 + 54 x + 52 x^2 + 14 x^3 + 13 x^4 + 70 x^5 + 55 x^6  \]
	
	\[
	\begin{array}{ccc}	
		
		s_1 = 56	& 		s_6 = 55	&		s_{11} = 24 \\
		s_2 = 62	&		s_7 = 46	&		s_{12} = 58 \\
		s_3 = 53	&		s_8 = 35	&		s_{13} = 6  \\
		s_4 = 29	&		s_9 = 64	&		s_{14} = 28 \\
		s_5 = 62	&		s_{10} = 39 &		s_{15} = 60 

	\end{array}		
	\]
	
	
	\[
	\begin{array}{ccccccc}	
		\omega_1 = 7 & \omega_2 = 52 & \omega_3 = 35 & \omega_4 = 38 & \omega_5 = 21 & \omega_6 = 66 & \omega_7 = 1
	\end{array}		
	\]
	
	\[
		s = \sum_{i=1}^{t} s_i \omega_i = 56\cdot 7 + 62 \cdot 52 + 53 \cdot 35 + 29 \cdot 38 + 62 \cdot 21 + 55 \cdot 66 + 46 \cdot 1 \mod p = 17
	\]
		
	\subsection{p=139, t=5, n=10}
	
	\[
	f(x) = 131 + 6 x + 77 x^2 + 127 x^3 + 63 x^4
	\]
	
	\[
\begin{array}{cc}	
	
	s_1 = 126	&	s_6 = 124 \\
	s_2 = 112	&	s_7 = 0	  \\
	s_3 = 61	&	s_8 = 0	  \\
	s_4 = 67	&	s_9 = 133 \\
	s_5 = 68	&	s_{10} = 113
	
\end{array}		
\]
	
	\[
	\begin{array}{ccccc}	
		\omega_1 = 5 & \omega_2 = 129 & \omega_3 = 10 & \omega_4 = 134 & \omega_5 = 1
	\end{array}		
	\]
	
	\[
		s = \sum_{i=1}^{t} s_i \omega_i = 126\cdot 5 + 112 \cdot 129 + 61 \cdot 10 + 67 \cdot 134 + 68 \cdot 1 \mod p = 131
	\]
	
	\newpage
	\stepcounter{section}
	\renewcommand\thesubsection{\Asbuk{subsection}}
	
	\section*{ПРИЛОЖЕНИЯ}
		\subsection{Текст программы} \label{program}
	
\lstinputlisting[language=Python]{shamir.py}
	
		
\end{document}