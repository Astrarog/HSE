\documentclass[a4paper,12pt]{article}

\usepackage{cmap}					% поиск в PDF
\usepackage[T2A]{fontenc}			% кодировка
\usepackage[utf8]{inputenc}			% кодировка исходного текста
\usepackage[english,russian]{babel}	% локализация и переносы
\usepackage{enumitem}  				% смена типа символя у enumerate
\usepackage{amsmath,amsfonts,amssymb,amsthm,epsfig,epstopdf,titling,url,array}
\usepackage{icomma} 				% "Умная" запятая: $0,2$ --- число, $0, 2$ --- перечисление
\usepackage{hyperref}				% кликабельные ссылки
\usepackage{soulutf8} 				% Модификаторы начертания
\usepackage{listings}				% Листинг исходного кода

\usepackage{xcolor}

\definecolor{codegreen}{rgb}{0,0.6,0}
\definecolor{codegray}{rgb}{0.5,0.5,0.5}
\definecolor{codepurple}{rgb}{0.58,0,0.82}
\definecolor{backcolour}{rgb}{0.95,0.95,0.92}

\lstdefinestyle{mystyle}{
	backgroundcolor=\color{backcolour},   
	commentstyle=\color{codegreen},
	keywordstyle=\color{magenta},
	numberstyle=\tiny\color{codegray},
	stringstyle=\color{codepurple},
	basicstyle=\ttfamily\footnotesize,
	breakatwhitespace=false,         
	breaklines=true,                 
	captionpos=b,                    
	keepspaces=true,                 
	numbers=left,                    
	numbersep=5pt,                  
	showspaces=false,                
	showstringspaces=false,
	showtabs=false,                  
	tabsize=2
}

\lstset{style=mystyle}

\allowdisplaybreaks

%\sloppy

\title{Реализация Схемы Блома распределения ключей }
\author{Роман Астраханцев, СКБ-171}

\begin{document}
	\maketitle
	
	\section{Написание программы}
	Была написана программа на языке программирования Python, которая вычисляет ключевые элементы по схеме Блома для $n$ участников, при которой обеспечивается стойкость к $m$ компрометациям, $1\le m \le n-2$. Величины $p, n, m$ задаются пользователем с консоли, $r_1=1, \dots, r_n=n$. Элементы симметричной матрицы коэффициентов $A$ исходного многочлена $f(x,y) = \sum_{i=0}^{m} \sum_{j=0}^{m} a_{i j} x^i y^j$ инициализируются случайно (как элементы $\mathbb{Z}_p$) при запуске программы. Текст программы может быть найден в приложении \ref{program}.
	
	\section{Расчёт ключевых значений}
	
	Студент выполнял вариант № k=4.

	\subsection{p=73, m=1, n=15}
	
	\[
	A = \begin{pmatrix}
		62 & 51 \\
		51 & 27 
	\end{pmatrix}
	\]
	
	\begin{align*}	
		g_1(x)  &= 40 + 5 x  \\
		g_2(x)  &= 18 + 32 x \\
		g_3(x)  &= 69 + 59 x \\
		g_4(x)  &= 47 + 13 x \\
		g_5(x)  &= 25 + 40 x \\
		g_6(x)  &= 3 + 67 x  \\
		g_7(x)  &= 54 + 21 x \\
		g_8(x)  &= 32 + 48 x \\
		g_9(x)  &= 10 + 2 x  \\
		g_{10}(x) &= 61 + 29 x \\
		g_{11}(x) &= 39 + 56 x \\
		g_{12}(x) &= 17 + 10 x \\
		g_{13}(x) &= 68 + 37 x \\
		g_{14}(x) &= 46 + 64 x \\
		g_{15}(x) &= 24 + 18 x
	\end{align*}	
		
	\subsection{p=139, m=2, n=10}
	
	\[
	A = \begin{pmatrix}
		116 & 131 & 35 \\
		131 & 84  & 83 \\
		35  & 83  & 26
	\end{pmatrix}
	\]
	
	\begin{align*}	
		g_1(x) &= 4 + 20 x + 5 x^2      \\
		g_2(x) &= 101 + 75 x + 27 x^2   \\
		g_3(x) &= 129 + 18 x + 101 x^2  \\
		g_4(x) &= 88 + 127 x + 88 x^2   \\
		g_5(x) &= 117 + 124 x + 127 x^2 \\
		g_6(x) &= 77 + 9 x + 79 x^2	    \\
		g_7(x) &= 107 + 60 x + 83 x^2   \\
		g_8(x) &= 68 + 138 x            \\
		g_9(x) &= 99 + 104 x + 108 x^2  \\
		g_{10}(x) &= 61 + 97 x + 129 x^2
	\end{align*}
	
	\newpage
	\stepcounter{section}
	\renewcommand\thesubsection{\Asbuk{subsection}}
	
	\section*{ПРИЛОЖЕНИЯ}
		\subsection{Текст программы} \label{program}
	
\lstinputlisting[language=Python]{blom.py}
	
		
\end{document}